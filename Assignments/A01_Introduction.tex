\documentclass[]{article}
\usepackage{lmodern}
\usepackage{amssymb,amsmath}
\usepackage{ifxetex,ifluatex}
\usepackage{fixltx2e} % provides \textsubscript
\ifnum 0\ifxetex 1\fi\ifluatex 1\fi=0 % if pdftex
  \usepackage[T1]{fontenc}
  \usepackage[utf8]{inputenc}
\else % if luatex or xelatex
  \ifxetex
    \usepackage{mathspec}
  \else
    \usepackage{fontspec}
  \fi
  \defaultfontfeatures{Ligatures=TeX,Scale=MatchLowercase}
\fi
% use upquote if available, for straight quotes in verbatim environments
\IfFileExists{upquote.sty}{\usepackage{upquote}}{}
% use microtype if available
\IfFileExists{microtype.sty}{%
\usepackage{microtype}
\UseMicrotypeSet[protrusion]{basicmath} % disable protrusion for tt fonts
}{}
\usepackage[margin=2.54cm]{geometry}
\usepackage{hyperref}
\hypersetup{unicode=true,
            pdftitle={Assignment 1: Introduction},
            pdfauthor={Keqi He},
            pdfborder={0 0 0},
            breaklinks=true}
\urlstyle{same}  % don't use monospace font for urls
\usepackage{graphicx,grffile}
\makeatletter
\def\maxwidth{\ifdim\Gin@nat@width>\linewidth\linewidth\else\Gin@nat@width\fi}
\def\maxheight{\ifdim\Gin@nat@height>\textheight\textheight\else\Gin@nat@height\fi}
\makeatother
% Scale images if necessary, so that they will not overflow the page
% margins by default, and it is still possible to overwrite the defaults
% using explicit options in \includegraphics[width, height, ...]{}
\setkeys{Gin}{width=\maxwidth,height=\maxheight,keepaspectratio}
\IfFileExists{parskip.sty}{%
\usepackage{parskip}
}{% else
\setlength{\parindent}{0pt}
\setlength{\parskip}{6pt plus 2pt minus 1pt}
}
\setlength{\emergencystretch}{3em}  % prevent overfull lines
\providecommand{\tightlist}{%
  \setlength{\itemsep}{0pt}\setlength{\parskip}{0pt}}
\setcounter{secnumdepth}{0}
% Redefines (sub)paragraphs to behave more like sections
\ifx\paragraph\undefined\else
\let\oldparagraph\paragraph
\renewcommand{\paragraph}[1]{\oldparagraph{#1}\mbox{}}
\fi
\ifx\subparagraph\undefined\else
\let\oldsubparagraph\subparagraph
\renewcommand{\subparagraph}[1]{\oldsubparagraph{#1}\mbox{}}
\fi

%%% Use protect on footnotes to avoid problems with footnotes in titles
\let\rmarkdownfootnote\footnote%
\def\footnote{\protect\rmarkdownfootnote}

%%% Change title format to be more compact
\usepackage{titling}

% Create subtitle command for use in maketitle
\providecommand{\subtitle}[1]{
  \posttitle{
    \begin{center}\large#1\end{center}
    }
}

\setlength{\droptitle}{-2em}

  \title{Assignment 1: Introduction}
    \pretitle{\vspace{\droptitle}\centering\huge}
  \posttitle{\par}
    \author{Keqi He}
    \preauthor{\centering\large\emph}
  \postauthor{\par}
    \date{}
    \predate{}\postdate{}
  

\begin{document}
\maketitle

\hypertarget{overview}{%
\subsection{OVERVIEW}\label{overview}}

This exercise accompanies the lessons in Hydrologic Data Analysis on
introductory material.

\hypertarget{directions}{%
\subsection{Directions}\label{directions}}

\begin{enumerate}
\def\labelenumi{\arabic{enumi}.}
\tightlist
\item
  Change ``Student Name'' on line 3 (above) with your name.
\item
  Work through the steps, \textbf{creating code and output} that fulfill
  each instruction.
\item
  Be sure to \textbf{answer the questions} in this assignment document
  (marked with \textgreater{}).
\item
  When you have completed the assignment, \textbf{Knit} the text and
  code into a single PDF file.
\item
  After Knitting, submit the completed exercise (PDF file) to the
  dropbox in Sakai. Add your last name into the file name (e.g.,
  ``FILENAME'') prior to submission.
\end{enumerate}

The completed exercise is due on 2019-09-04 before class begins.

\hypertarget{course-setup}{%
\subsection{Course Setup}\label{course-setup}}

\begin{enumerate}
\def\labelenumi{\arabic{enumi}.}
\tightlist
\item
  Post the link to your forked GitHub repository below. Your repo should
  include one or more commits and an edited README file.
\end{enumerate}

\begin{quote}
Link:
\end{quote}

\begin{enumerate}
\def\labelenumi{\arabic{enumi}.}
\setcounter{enumi}{1}
\tightlist
\item
  Complete the Consent Form in Sakai. You must choose to either opt in
  or out of the research study being conducted in our course.
\end{enumerate}

Did you complete the form? (yes/no)

\begin{quote}
yes
\end{quote}

\hypertarget{course-project}{%
\subsection{Course Project}\label{course-project}}

\begin{enumerate}
\def\labelenumi{\arabic{enumi}.}
\setcounter{enumi}{2}
\tightlist
\item
  What are some topics in aquatic science that are particularly
  interesting to you?
\end{enumerate}

\begin{quote}
ANSWER:
\end{quote}

\begin{enumerate}
\def\labelenumi{\arabic{enumi}.}
\setcounter{enumi}{3}
\tightlist
\item
  Are there specific people in class who you would specifically like to
  have on your team?
\end{enumerate}

\begin{quote}
ANSWER:
\end{quote}

\begin{enumerate}
\def\labelenumi{\arabic{enumi}.}
\setcounter{enumi}{4}
\tightlist
\item
  Are there specific people in class who you would specifically
  \emph{not} like to have on your team?
\end{enumerate}

\begin{quote}
ANSWER:
\end{quote}

\hypertarget{data-visualization-exercises}{%
\subsection{Data Visualization
Exercises}\label{data-visualization-exercises}}

\begin{enumerate}
\def\labelenumi{\arabic{enumi}.}
\setcounter{enumi}{5}
\item
  Set up your work session. Check your working directory, load packages
  \texttt{tidyverse}, \texttt{dataRetrieval}, and \texttt{lubridate}.
  Set your ggplot theme as theme\_classic (you may need to look up how
  to set your theme).
\item
  Upload discharge data for the Eno River at site 02096500 for the same
  dates as we studied in class (2009-08-01 through 2019-07-31). Obtain
  data for discharge and gage height (you will need to look up these
  parameter codes). Rename the columns with informative titles. Imperial
  units can be retained (no need to change to metric).
\item
  Add a ``year'' column to your data frame (hint: lubridate has a
  \texttt{year} function).
\item
  Create a ggplot of discharge vs.~gage height, with gage height as the
  x axis. Color each point by year. Make the following edits to follow
  good data visualization practices:
\end{enumerate}

\begin{itemize}
\tightlist
\item
  Edit axes with units
\item
  Change color palette from ggplot default
\item
  Make points 50 \% transparent
\end{itemize}

\begin{enumerate}
\def\labelenumi{\arabic{enumi}.}
\setcounter{enumi}{9}
\tightlist
\item
  Interpret the graph you made. Write 2-3 sentences communicating the
  main takeaway points.
\end{enumerate}

\begin{quote}
ANSWER:
\end{quote}

\begin{enumerate}
\def\labelenumi{\arabic{enumi}.}
\setcounter{enumi}{10}
\tightlist
\item
  Create a ggplot violin plot of discharge, divided by year. (Hint: in
  your aesthetics, specify year as a factor rather than a continuous
  variable). Make the following edits to follow good data visualization
  practices:
\end{enumerate}

\begin{itemize}
\tightlist
\item
  Remove x axis label
\item
  Add a horizontal line at the 0.5 quantile within each violin (hint:
  draw\_quantiles)
\end{itemize}

\begin{enumerate}
\def\labelenumi{\arabic{enumi}.}
\setcounter{enumi}{11}
\tightlist
\item
  Interpret the graph you made. Write 2-3 sentences communicating the
  main takeaway points.
\end{enumerate}

\begin{quote}
ANSWER:
\end{quote}


\end{document}
